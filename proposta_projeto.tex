% !TEX program = xelatex
\documentclass[12pt,brazilian]{article}
\input{r/preamble.tex}
\usepackage{epsf}
\usepackage{floatflt}
\usepackage{float}
\usepackage{accents}
\usepackage{multicol}
\usepackage{upgreek}
% Comando para derivadas temporais
\newcommand*{\dt}[1]{
    \dv{#1}{t}
    }
\newcommand*{\ddt}[1]{
    \dv[2]{#1}{t}
    }


\tipo{Proposta de Projeto}
\titulo{Caracterização de Motores}
\nucleo{Núcleo de Engenharia}


\date{\today}



\begin{document}
\setTema
\geraTituloProjeto
\sumario

% DOCUMENTO COMEÇA AQUI

\section{Descrição do Projeto}

\lipsum[3]

\section{Modelo do Motor}

\noindent
O circuito correspondente do motor é conforme a \Cref{fig:circ_motor}.

\addFigure{0.4}{circ_motor}{Modelo do Motor}

Equação de Tensão-Velocidade Angular no motor:
\begin{equation}
    V_M(t) = K_R\cdot\omega(t)=K_R \dt{\theta(t)}  .
\end{equation}
Lei das malhas no circuito:
\begin{equation*}
    \sum V = 0 \Rightarrow V(t) = V_R(t) + V_L(t) + V_M(t)
\end{equation*}
\begin{equation}\label{eq:EquacaoDiferencialTensao}
    V(t) - R i(t) - L\dt{i(t)} - K_R\dt{\theta(t)} = 0 .
\end{equation}
Torque gerado pelo motor:
\begin{equation}
    \uptau(t) = K_M\cdot i(t) .
\end{equation}
Conservação de momento angular:
\begin{equation*}
    \dt{\vec{L}} = J\vec{\alpha}= \sum \vec{\uptau} \Rightarrow J\ddt{\theta(t)} = \uptau(t) -b \dt{\theta(t)}
\end{equation*}
\begin{equation}\label{eq:EquacaoDiferencialTorque}
    J\ddt{\theta(t)} = K_M i(t) -b \dt{\theta(t)},
\end{equation}
onde $b$ é um fator de atrito.


\section{Equação de Estado Sistema}

O motor elétrico pode, então, ser definido pelas seguintes equações diferenciais:
\begin{equation}
\arraycolsep=1.4pt\def\arraystretch{2.2}
\left\{\begin{array}{l}
{ \dv{t}\theta(t)=\omega(t)} \\ 
{J \dv{t}\omega(t)= K_M i(t)} - b \omega(t) \\ 
{L \dv{t}i(t)+R i(t)+K_R\omega(t)=V(t)} 
\end{array}\right.
\end{equation}

Podemos, então, escrever na forma matricial

\begin{equation}
    \dv{t} \mqty[\theta(t) \\ \omega(t) \\ i(t)]=
    \mqty[ 0 & 1 & 0 \\ 0 & -b/J & K_M/J \\ 0 & -K_R/L & -R/L] 
    \mqty[\theta(t) \\ \omega(t) \\ i(t)]
    + \mqty[0\\0\\\frac{1}{L}]
    V(t)
\end{equation}




\section{Levantamento de Requisitos}
 Os requisitos do sistemas foram levantados pensando-se na caracterização de hélices e de motores do tipo brushed e brushless.
 Pensando-se na caracterização de motores, os seguintes requisitos foram levantados:
 \begin{itemize}
    \item Resistência interna do motor a ser caracterizado
    \item Indutância do motor a ser caracterizado
    \item Corrente em ambos os motores
    \item Tensão no motor conhecido
    \item Tempo de parada do motor a ser caracterizado
    \item Medição de corrente de Stoll
    \item Fonte de tensão com saída ajustável por software para motor a ser caracterizado
    \item Velocidade angular dos motores
    \item Posição angular
    \item Acoplador entre dois eixos
    \item Percas no motor elétricos
 \end{itemize}
 
 Já para a caracterização de hélices, os requisitos levantados foram os seguintes:
 \begin{itemize}
     \item Dinamômetro para medição da força da hélice
     \item Velocidade angular da hélice
     \item Velocidade de entrada e de saída do vento
     \item Pressão de entrada e de saída do vento
     \item Controle de velocidade da hélice
     \item Estrutura aerodinâmica 
     \item Trilho com baixo coeficiente de atrito 
     \item Estrutura resistente a vibrações
 \end{itemize}
\end{document}
